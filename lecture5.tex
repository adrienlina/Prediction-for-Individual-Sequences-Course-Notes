\documentclass{article}

\usepackage{fontenc}

\usepackage[french]{babel}

\usepackage[utf8]{inputenc}

\usepackage{graphicx}

\usepackage{amsmath}
  \DeclareMathOperator*{\argmin}{argmin}

  \DeclareMathOperator*{\argmax}{argmax}

  \DeclareMathOperator*{\supp}{supp}

  \DeclareMathOperator*{\E}{\mathbb{E}}

  \let\P\relax
  \DeclareMathOperator*{\P}{\mathbb{P}}

  \DeclareMathOperator*{\R}{\mathbb{R}}

\usepackage{amssymb}

\usepackage{amsthm}
  \newtheorem{definition}{Définition}[section]
  \newcommand{\definitionautorefname}{Définition}

  \newtheorem{theorem}{Théorème}[section]
  \let\theoremautorefname\relax
  \newcommand{\theoremautorefname}{Théorème}

  \newtheorem{proposition}{Proposition}[section]
  \newcommand{\propositionautorefname}{Proposition}

  \newtheorem{hypothesis}{Hypothèse}[section]
  \newcommand{\hypothesisautorefname}{Hypothèse}

  \newtheorem{lemma}{Lemme}[section]
  \newcommand{\lemmaautorefname}{Lemme}

  \theoremstyle{remark}
  \newtheorem{example}{Exemple}[section]

  \theoremstyle{remark}
  \newtheorem{remark}{Remarque}[section]

\usepackage{bbold}

\usepackage{graphicx}

\usepackage{hyperref}


\begin{document}

\title{%
  \large MVA ENS Cachan Paris Saclay \\
  \huge Prediction for Individual Sequences\\ Notes de Cours \\
}
\author{%
  Cours donné par Vianney Perchet \\
  Note prises par Adrien Lina
}

\maketitle

\section{Lecture 3 : Bandits contextuels}

\subsection{Rappels}

\textbf{Cadre}: pour chaque $t \geq 1$,
\begin{itemize}
   \item le joueur choisit $p_t \in \Delta_K = \{p \in [0,1]^K : \sum p^{(k)} = 1\}$
   \item on observe
   \begin{itemize}
      \item son reward $X_t^{(k)}$ "bandit"
      \item $X_t^{(1)}, ..., X_t^{(K)}$ "info complête"
   \end{itemize}
\end{itemize}

\textbf{But} : minimiser le regret $R_T^{(k)} = \sum_{t=1}^T X_t^{(k)} - \sum_{t=1}^T X_t^{(\pi_t)}$

On a vu jusqu'ici comment minimiser le pseudoregret $\max_k \E[R_T^{(k)}]$.

On va maintenant voir comment minimiser $\max_k R_T^{(k)}$.

\subsection{EXP3.P}

\begin{definition}[EXP3.P]
   On choisit chaque bras avec une probabilité :
   $$
   p_t^{(k)} = (1-\gamma) \frac{e^{\eta \widehat{R_T^{(k)}}}}{\sum\limits_{j=1}^Ke^{\eta \widehat{R_T^{(j)}}}} + \frac{\gamma}{K}
   $$
   où
   \begin{align*}
      \widehat{R_T^{(k)}} &= \sum_{s=1}^{t-1} \left( \widehat{X_s^{(k)}} - \widehat{X_s^{(\pi_s)}} \right) \\
      \widehat{X_t^{(k)}} &= X_t^{(j)} \mathbb{1}_{\{k = \pi_t\}} + \beta
   \end{align*}
\end{definition}

\begin{theorem}
   EXP3P vérifie :
   $$
   R_T \leq T \sqrt{T K \log K} + \sqrt{\frac{T K}{\log K}} \log\left(\frac{1}{\delta}\right)
   $$
   pour $\eta > 0$, $\gamma \in [0,1]$ et $\beta \in [0,1]$ choisis tels que
   $$
   \eta (1+\beta) \frac{K}{\gamma} \leq 1
   $$
\end{theorem}

\begin{lemma}
   pour $\beta \in [0,1]$,
   $$
   \sum_{t=1}^T \widehat{X_t^{(k)}} > \sum_{t=1}^T X_t^{(k)} - \frac{\log(1/\delta)}{\beta}
   $$
\end{lemma}

\begin{proof}
   On pose $q_t^{(k)} = \frac{e^{\eta \widehat{R_T^{(k)}}}}{\sum\limits_{j=1}^Ke^{\eta \widehat{R_T^{(j)}}}}$.

   D'après la preuve de EXP, on a
   \begin{align}
      \max_k \sum_{t=1}^T \widehat{X_t^{(k)}} - \sum_{t=1}^T \mathbb{E}_{j \sim q_t}\left[\widehat{X_t^{(j)}}\right]
      &\leq \frac{\log K}{\eta} + \eta \sum_{t=1}^T \mathbb{E}_{j \sim q_t}\left[\left(\widehat{X_t^{(j)}}\right)^2\right]
      \label{eq:EXP}
   \end{align}
   avec
   $$
   \mathbb{E}_{j \sim q_t}\left[\widehat{X_t^{(j)}}\right] = \sum_{j=1}^K q_t^{(j)} \widehat{X_t^{(j)}}
   $$

   On calcule:
   \begin{align*}
      \mathbb{E}_{j \sim q_t}\left[\widehat{X_t^{(j)}}\right] &= \sum_{j=1}^K q_t^{(j)} \widehat{X_t^{(j)}} \\
      &\leq \frac{1}{1-\gamma} \sum_{j=1}^K p_t^{(j)} \widehat{X_t^{(j)}}
   \end{align*}
   car $p_t^{(k)} = (1-\gamma) q_t^{(k)} + \frac{\gamma}{K}$ et donc $q_t^{(k)} \leq \frac{p_t^{(k)}}{1-\gamma}$.

   Donc
   \begin{align*}
      \mathbb{E}_{j \sim q_t}\left[\widehat{X_t^{(j)}}\right] &= \sum_{j=1}^K q_t^{(j)} \widehat{X_t^{(j)}} \\
      &\leq \frac{1}{1-\gamma} \sum_{j=1}^K p_t^{(j)} \frac{X_t^{(j)} \mathbb{1}_{\{k=\pi_t\}} + \beta}{p_t^{(j)}} \\
      &\leq  \frac{1}{1-\gamma} \left(X_t^{\pi_t} + K \beta\right)
   \end{align*}

   De même,
   \begin{align*}
      \mathbb{E}_{j \sim q_t}\left[\left(\widehat{X_t^{(j)}}\right)^2\right] &\leq \frac{1}{1 - \gamma} \sum_{k=1}^K p_t^{(k)} \left(\widehat{X_t^{(k)}}\right)^2 \\
      &\leq \frac{1}{1-\gamma} \sum_{k=1}^K \widehat{X_t^{(k)}} (1+\beta)
   \end{align*}
   car $p_t^{(k)}\widehat{X_t^{(k)}} = X_t^{(k)} \mathbb{1}_{\{k = \pi_t\}} + \beta \leq 1 + \beta$

   Donc
   \begin{align*}
     \sum_{t=1}^T \mathbb{E}_{j \sim q_t}\left[\left(\widehat{X_t^{(j)}}\right)^2\right] &\leq \frac{1 + \beta}{1 - \gamma} \sum_{t=1}^T \sum_{k=1}^K \widehat{X_t^{(k)}} \\
      &\leq \frac{1 + \beta}{1 - \gamma} \sum_{k=1}^K \sum_{t=1}^T \widehat{X_t^{(k)}} \\
      &\leq \frac{(1 + \beta) K}{1 - \gamma} \max_k \left( \sum_{t=1}^T \widehat{X_t^{(k)}} \right)
   \end{align*}

   En remplaçant dans \eqref{eq:EXP} :
   \begin{align*}
      \max_k \sum_{t=1}^T \widehat{X_t^{(k)}} - \frac{1}{1-\gamma} \left( \sum_{t=1}^T X_t^{\pi_t} + T K \beta \right) &\leq \frac{\log K}{\eta} + \eta \frac{(1 + \beta) K}{1 - \gamma} \max_k \left( \sum_{t=1}^T \widehat{X_t^{(k)}} \right) \\
   \end{align*}
   i.e.
   \begin{align*}
      (1-\gamma) \max_k \left( \sum_{t=1}^T \widehat{X_t^{(k)}} \right) &\leq \sum_{t=1}^T X_t^{(\pi_k)} + T K \beta + \frac{1-\gamma}{\eta} \log K + \underbrace{\eta (1+\beta) K }_{\leq \gamma \text{ par hypothèse}} \max_k \sum_{t=1}^T \widehat{X_t^{(k)}}
   \end{align*}
   Donc
   \begin{align*}
      (1-2\gamma) \max_k \left( \sum_{t=1}^T \widehat{X_t^{(k)}} \right) &\leq \sum_{t=1}^T X_t^{(\pi_k)} + T K \beta + \frac{1-\gamma}{\eta} \log K
   \end{align*}

   Or d'après le lemme
   \begin{align*}
      \P\left( \max_k \left( \sum_{t=1}^T \widehat{X_t^{(k)}} \right) \leq \max_k \left( \sum_{t=1}^T X_t^{(k)} \right) - \frac{\log \frac K \delta}{\beta} \right) &= \P\left( \exists k :  \sum_{t=1}^T \widehat{X_t^{(k)}} \leq  \sum_{t=1}^T X_t^{(k)} - \frac{\log \frac K \delta}{\beta} \right) \\
      &\leq \sum_{k=1}^K \P\left( \sum_{t=1}^T \widehat{X_t^{(k)}} \leq  \sum_{t=1}^T X_t^{(k)} - \frac{\log \frac K \delta}{\beta} \right) \\
      &\leq \sum_{k=1}^K \frac \delta K \\
      &\leq \delta
   \end{align*}

   Donc avec proba au moins $1-\delta$:
   \begin{align*}
      (1-2\gamma) \max_k \left( \sum_{t=1}^T \widehat{X_t^{(k)}} \right) &\leq \sum_{t=1}^T X_t^{(\pi_k)} + T K \beta + \frac{\log K}{\eta}
   \end{align*}
   En reorganisant:
   \begin{align*}
      R_t = \max_k \sum_{t=1}^T X_t^{(k)} - \sum_{t=1}^T X_t^{(\pi_t)} &\leq T K \beta + \frac{\log K}{\eta} + 2 \gamma \underbrace{\max_k \left( \sum_{t=1}^T X_t^{(k)} \right)}_{\leq T} + \underbrace{(1-2 \gamma)}_{\leq 1} \frac{\log \frac K \delta}{\beta} \\
      &\leq T K \beta + \frac{\log \frac K \delta}{\beta} + \frac{\log K}{\eta} + 2 \gamma T
   \end{align*}

   On choisit $\gamma = 2 \eta K$ de sorte que $\eta (1+\beta) \frac K \gamma \leq 1$ pour $\beta \in [0,1]$:
   $$
   R_T \leq T K \beta + \frac{\log \frac K \delta}{\beta} + \frac{\log K}{\eta} + 4 \eta K T
   $$

   On choisit $\eta$ tel que : $\frac{\log K} \eta = 4 \eta K T$, donc $\eta = \frac 1 2 \sqrt{\frac{\log K}{K T}}$ e

   On choisit $\beta$ tel que $T K \beta = \frac{\log K}{\beta}$, donc $\beta = \sqrt{\frac{\log K}{K T}}$.

   En remplaçant dans la borne:
   $$
   R_T \leq \gamma \sqrt{T K \log K} + \sqrt{\frac{T K}{\log K}} \log \frac 1 \delta
   $$
\end{proof}

\begin{remark}
   Les algos précédents EXP, EXP3 et EXP3.P dépendent de paramètres $\eta$, $\beta$ et $\gamma$ qui ont été optimisés en fonction de $T$.

   Les résultats qu'on a vu ne sont donc valabes que pour un horizon $T$ connu à l'avance.

   Comment faire pour tout $t$ ?
\end{remark}

\begin{definition}[Doubling Trick]
   À chaque fois que $t$ est une puissance de 2, on redémarre l'algorithme avec $\eta = \sqrt{\frac{\log K}{K T}}$ eb oubliant tout ce qui a été appris.
\end{definition}

\begin{theorem}
   Avec EXP3 et le doubling Trick, on obtient:
   $$
   \max_k \E[R_T^{(k)}] \leq 7 \sqrt{T K \log K }
   $$
\end{theorem}

\begin{remark}
   On a juste perdu un facteur multiplicatif.
\end{remark}

\begin{proof}
   Si $2^M \leq T < 2^{M+1}$, $T$ inconnu.

   Le pseudo-regret de EXP3 + Doubling Trick devient
   \begin{align*}
      \max_k \E[R_T^{(k)}] &= \max_k \E\left[\sum_{t=1}^T X_t^{(k)} - X_t^{(\pi_t)} \right] \\
      &\leq \max_k \E\left[\sum_{m=0}^M  \sum_{t=2^m}^{2^{m+1}} X_t^{(k)} - X_t^{(\pi_t)} \right] \\
      &\leq \sum_{m=0}^M  \underbrace{\max_k \E\left[ \sum_{t=2^m}^{2^{m+1}} X_t^{(k)} - X_t^{(\pi_t)} \right]}_{\text{EXP3 de paramètre $\eta = \sqrt{\frac{\log K}{K 2^m}}$}} \\
      &\leq \sum_{m=0}^M 2 \sqrt{2^m K \log K} \\
      &\leq 2 (1+\sqrt{2}) \sqrt{2^{M+1} K \log K} \\
      &\leq 7 \sqrt{T K \log K}
   \end{align*}
\end{proof}

\begin{theorem}
   Le doubling trick fonctionne pour n'importe quel algorithme sequentiel en regret $\mathcal{O}(T^a)$ pour le transformer en un algo de $\mathcal{O}(T^a)$ pour tout T.
\end{theorem}

\begin{remark}
   En pratique, c'est très mauvais.
   On préfère utiliser des paramètres $\eta_t$ où $T$ est remplacé par $t$.
   Théoriquement, on peut prouver que cela fonctionne, mais c'est plus laborieux.
\end{remark}

\subsection{Optimalité des algos}

On a vu des algos qui ont un regret
\begin{itemize}
   \item en $\mathcal{O}(\sqrt{T \log K})$ en info complète ;
   \item en $\mathcal{O}(\sqrt{T K \log K})$ en bandit ;
\end{itemize}

Peut-on faire mieux ?

\begin{theorem}
   En info complète, pour tout algo $R_t \geq \text{const} \times \sqrt{T \log K}$.
   En bandit, pour tout algo $R_t \geq \text{const} \times \sqrt{T K}$.
\end{theorem}

\begin{proof}[Intuition]
   \begin{itemize}
      \item Pour l'info complète:

      Considérons un adversaire qui choisit $X_t^{(k)} \sim \text{Ber}(\frac 1 2)$ iid.

      \begin{align*}
         \E[\sum_{t=1}^T X_t^{(k)}] &= \frac T 2 \\
         \E[\max_k \sum_{t=1}^T X_t^{(k)}] &= \E[\max \{\text{$K$ marches aléatoires}\}] \\
         &\simeq \frac T 2 + \sqrt{T \log K}
      \end{align*}
      \item Pour le bandit :

      L'adversaire ne choisit que des $\text{Ber}(\frac 1 2)$ sauf un bras $k^*$ suivant $\text{Ber}(\frac 1 2 + \epsilon)$.
      Ainsi, pour différencier les bras, il faut tirer $\epsilon^{-2}$ observations de chaque bras.

      En effet, d'après le théorème limite central, après $T_i$ observations d'une variable aléatoire, on peut estimer son espérance avec une erreur de l'ordre de $\frac{1}{\sqrt{T_i}}$.

      Un des bras est tiré moins de $\frac T K$ fois. Si c'est le bras $k^*$, on ne pourra se rendre compte que c'est le meilleur (normalement à partir de $\frac T K \leq \epsilon^{-2}$).

      On aura tiré les autres bras au moins $T - \frac T K = (1- \frac 1 K) T$ fois. On a donc un regret pour $\epsilon = \sqrt{\frac T K}$ de $(1- \frac 1 K) T \epsilon \approx (1 - \frac 1 K) \sqrt{T K}$.
   \end{itemize}
\end{proof}

\subsection{Online learning with expertise}

On a plusieurs algorithmes / experts qui nous proposent des avis à chaque temps $t$. On veut se rapprocher de la performance du meilleur expert.

\begin{definition}[Online learning with expertise]
   $\quad$ % Trick to avoid strange paragraphing

   \textbf{Cadre}: à chaque temps $t$,
   \begin{itemize}
      \item $N$ expertes proposent des prévisions $\xi_t^{(i)} \in \mathcal{A}$ pour $i = 1,...,N$ ;
      \item On fait notre prévision $\pi_t \in \mathcal{A}$ et l'adversaire choisit une fonction de gain $g_t : \mathcal{A} \rightarrow [0,1]$ ;
      \item Le joueur observe
      \begin{itemize}
         \item $g_t$ en info complète ;
         \item $g_t(\pi_t)$ en bandit.
      \end{itemize}
   \end{itemize}

   \textbf{But}: minimiser le regret par rapport au experts :
   $$
   R_T = \max_k \sum_{t=1}^T g_t(\xi_t^{(k)}) - g_t(\pi_t)
   $$
\end{definition}

\begin{remark}
   Si $\mathcal{A} = \{1,...,N\}$ et $\xi_t^{(i)} = i$ et $g_t(i) = X_t^{(i)}$, on retrouve le cardre de bandit traditionnel. Cependant, ce cadre est plus riche : on peut par exemple considérer des ensemble $\mathcal{A}$ continus.
\end{remark}

\begin{remark}
   en appliquant les algos précédants en considérant es experts come des bras avec $X_t^{(k)} = g_t(\xi_t^{(k)})$, les algos précédants sont valables ici.

   \begin{itemize}
      \item En info complète, $R_T \leq \sqrt{T \log N}$.
      \item En bandit, $R_T \leq \sqrt{T N \log N}$.
   \end{itemize}

   Le but est d'avoir de meilleurs résultats ici.
\end{remark}

\subsubsection{Info complète avec gains exp-concaves}

L'idée est la suivante : on a une suite $y_1, ..., y_T$ choisit par un adversaire à prévoir, $y_i \in \mathcal{Y} = \R^d$. À chaque temps $t$, chaque expert essaye de prévoir $y_t$ avec $\xi_t^{(i)} \in \R^d$ et l'adversaire révèle les gains $g_t(\xi) = g(\xi, y_t)$ qui augmente quand $\xi$ et $y_t$ sont proches.

\begin{example}
   On peut avoir $g_t(\xi, y_t) = - || \xi - y_t||^2$.
\end{example}

\begin{definition}[Exp-concavité]
   $g$ est $\eta$-exp-concave si
   $\forall y \in \mathcal{Y}, \xi \mapsto e^{\eta g(\xi, y)}$ est concave.
\end{definition}

\begin{remark}
   L'exp-concavité est plus forte que la concavité en le premier argument.
\end{remark}

\begin{example}
   Quelques expamples de fonctions exp-concaves :
   \begin{itemize}
      \item $g : (\xi, y) \in [0,1]^2  \mapsto - (\xi - y)^2$ est $(\frac 1 2)$-exp-concave.

         En effet, en notant $G_y : \xi \mapsto g(\xi,y)$, et en prennant $y \in [0,1]$, on a :

         $$
         G_y''(\xi) = (4 \eta^2 (\xi - y)^2 - 2 \eta) e^{-\eta(\xi - y)^2}
         $$
         Donc
         \begin{align*}
            G_y''(\xi)  \leq 0 & \Leftrightarrow 4 \eta^2 (\xi - y)^2 - 2 \eta \leq 0  \\
            & \Leftarrow 2 \eta \underbrace{(\xi - y)^2}_{\leq 1} \leq 1 \\
            & \Leftarrow \eta \leq \frac 1 2
         \end{align*}
      \item De même, $g : (\xi, y) \in [0,1]^{d \times d}  \mapsto - ||\xi - y||^2$ est $(\frac 1 2)$-exp-concave.
      \item Le gain induit par l'entropie relative est 1-exp-concave :
         $$
            g: (\xi, y) \in [0,1]^2 \mapsto -y \log\left(\frac y \xi \right) - (1-y) \log\left(\frac{1-y}{1-\xi}\right)
         $$
   \end{itemize}

   Quelques exemples de fonction qui ne sont pas $\eta$-exp-concave pour $\eta > 0$ :
   \begin{itemize}
      \item $g: (\xi, y) \mapsto -|\xi-y|$
      \item $g: (\xi, y) \mapsto \langle \xi,y \rangle$
   \end{itemize}
\end{example}

\begin{definition}[EXP]
   On assigne le poids
   $$
   P_t^{(k)} = \frac{e^{\eta R_{t-1}^{(i)}}}{\sum\limits_{j=1}^Ke^{\eta R_{t-1}^{(j)}}}
   $$
   à chacun des experts $i = 1,...,N$ et on choisit
   $$
   \pi_t = \mathbb{E}_{j \sim p_t}[\xi_t^{(j)}] = \sum_{j=1}^N p_t^{(j)} \xi_t^{(j)}
   $$
   où
   $$
   R_t^{(i)} = \sum_{s=1}^{t-1} g_s(\xi_s^{(i)}) - g_s(\pi_s)
   $$
\end{definition}

\begin{theorem}
   Si
   \begin{itemize}
      \item $\mathcal{A}$ est convexe ;
      \item $g_t(\xi) = g(\xi, y_t)$ ;
      \item $g$ est $\eta$-exp-convexe, $\eta > 0$.
   \end{itemize}

   Alors EXP utilisé avec le paramètre a un regret $R_T \leq \frac{\log N}{\eta}$.
\end{theorem}

\begin{proof}
   (C'est presque la même que celle de EXP)

   On note $W_t^{(i)} = e^{\eta \sum\limits_{s=1}^{t-1} g_s(\xi_s^{(i)})}$ et $W_t = \sum_{i=1}^N W_t^{(i)}$.

   On majore et on minore $W_t$ :
   \begin{align*}
      W_t &= \sum_{i=1}^N W_t^{(i)} \\
      &= \sum_{i=1}^N W_{t-1}^{(i)} e^{\eta g_t(\xi_t^{(i)})} \\
      &= W_{t-1} \sum_{i =1}^N \underbrace{\frac{W_{t-1}^{(i)}}{W_{t-1}}}_{p_t^{(i)}} e^{\eta g_t(\xi_t^{(i)})}
      &= W_{t-1} \sum_{i =1}^N p_t^{(i)} e^{\eta g_t(\xi_t^{(i)})} \\
      &\leq W_{t-1} \exp\Bigg( \eta g\Big(\underbrace{\sum_{i=1}^N p_t^{(i)} \xi_t^{(i)}}_{= \pi_t}, y_t\Big) \Bigg) \\
      &= W_{t-1} \exp(\eta g_t(\pi_t))
   \end{align*}

   Par induction, et comme $W_0 = N$, on a : $W_t \leq N \exp\left(\eta \sum\limits_{t=1}^T g_t(\pi_t) \right)$

   On minore $W_t$:
   $$
   W_t = \sum_{t=1}^T W_t^{(i)} \geq \max_i e^{\eta \sum\limits_{t=1}^T g_t(\xi_t^{(i)})} = e^{\eta \max\limits_i \sum\limits_{t=1}^T g_t(\xi_t^{(i)})}
   $$
   En combinant et en prenant le log, on a : $R_t = \max\limits_i \sum\limits_{t=1}^T g_t(\xi_t^{(i)})  - g_t(\pi_t) \leq \frac{\log N}{\eta}$
\end{proof}

\subsubsection{Bandit avec experts}

Les experts $\xi_t^{(i)} \in \{1, ..., K\}$ pour $i=1,...,N$ et les gains $g_t(\xi) = X_t^{(i)}$.

Appliquer maintenant EXP3 sur les experts donne un regret en $\sqrt{T N \log N}$. On peut faire mieux si $N>K$ car on n'utilise pas le fait que quand on choisit le bras $\pi_t \in \{1,...,K\}$ on observe la performace de tous les experts tels qu $\xi_t^{(i)} = \pi_t$.

\begin{definition}[EXP4]
   A l'instant $t$:
   \begin{itemize}
      \item Observer les avis d'experts $\xi_t^{(i)} \in \{1, ..., K\}$
      \item Choisir l'action $\pi_t = \xi_t^{(i)}$ avec proba $q_t^{(i)}$
      \item Observer le gain $X_t^{(\pi_t)}$
      \item Estimer le gain de chacun des bras : $\widehat{X_t^(k)} = \frac{X_t^{(k)}-1}{p_t^{(k)}} \mathbb{1}_{\{k = \pi_t\}}$
      \item Mettre à jour les proba des experts
      $$
         \widehat{R_{t-1}} = \sum_{s=1}^{t-1} \widehat{X_t^{(k)}} - \widehat{X_t^{(\pi_s)}}
      $$
   \end{itemize}
\end{definition}

\begin{theorem}
   Le pseudoregret par rapport au meilleur expert
   $$
   R_t = \max_i \E[]
   $$
\end{theorem}

\end{document}
